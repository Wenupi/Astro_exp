% Template:     Informe LaTeX
% Documento:    Archivo principal
% Versión:      7.0.3 (02/08/2020)
% Codificación: UTF-8
%
% Autor: Pablo Pizarro R.
%        Facultad de Ciencias Físicas y Matemáticas
%        Universidad de Chile
%        pablo@ppizarror.com
%
% Manual template: [https://latex.ppizarror.com/informe]
% Licencia MIT:    [https://opensource.org/licenses/MIT]

% CREACIÓN DEL DOCUMENTO
\documentclass[letterpaper,oneside]{article}

% INFORMACIÓN DEL DOCUMENTO
\def\titulodelinforme {Cinemática Galáctica}
\def\temaatratar {Informe Tarea 2}

\def\autordeldocumento {Javier Huenupi}
\def\nombredelcurso {Astronomía experimental}
\def\codigodelcurso {AS3201-1 }

\def\nombreuniversidad {Universidad de Chile}
\def\nombrefacultad {Facultad de Ciencias Físicas y Matemáticas}
\def\departamentouniversidad {Departamento de Astronomía}
\def\imagendepartamento {departamentos/das}
\def\imagendepartamentoescala {0.2}
\def\localizacionuniversidad {Santiago, Chile}

% INTEGRANTES, PROFESORES Y FECHAS
\def\tablaintegrantes {
\begin{tabular}{ll}
	Integrantes:
	& \begin{tabular}[t]{l}
		Javier Huenupi
	\end{tabular} \\
	Profesor:
	& \begin{tabular}[t]{l}
		Edgardo Costa\\
		Leonardo Bronfman
	\end{tabular} \\
	Auxiliar:
	& \begin{tabular}[t]{l}
		Paulina Palma
	\end{tabular} \\

	\multicolumn{2}{l}{Fecha de realización: \today} \\
	\multicolumn{2}{l}{Fecha de entrega: \today} \\
	\multicolumn{2}{l}{\localizacionuniversidad}
\end{tabular}}{
}

% IMPORTACIÓN DEL TEMPLATE
\input{template}

% INICIO DE PÁGINAS
\begin{document}
	
% PORTADA
\templatePortrait

% CONFIGURACIÓN DE PÁGINA Y ENCABEZADOS
\templatePagecfg

% RESUMEN O ABSTRACT
%\begin{resumen}
%	\lipsum[1] % Párrafo ejemplo, se puede borrar
%\end{resumen}

% TABLA DE CONTENIDOS - ÍNDICE
\templateIndex

% CONFIGURACIONES FINALES
\templateFinalcfg

% ======================= INICIO DEL DOCUMENTO =======================

%\input{example} % Ejemplo, se puede borrar
\section{Introducción}
Se estudia el cuarto cuadrante

\section{Curva de rotación}
% https://en.wikipedia.org/wiki/Galaxy_rotation_curve#:~:text=The%20rotation%20curve%20of%20a,distance%20from%20that%20galaxy's%20centre.&text=Stars%20revolve%20around%20their%20galaxy's,a%20large%20range%20of%20distances.

%La velocidad depende de la distancia al centro

%Constantes para el Sol de la IAU (no están actualizadas)
%\begin{gather*}
%{v_{\odot}}^{rot}=220 \ km/s\\
%R_{\odot} = 8.5 \ kpc
%\end{gather*}
%En \href{https://www.youtube.com/watch?v=Ijj2VhVSFsg}{la clase} se llega a la \textit{ecuación maestra}, pero para el caso en el que el objeto es tangencial. También se demuestra que $\omega$ disminuye con el radio.

%Parece que se ocupa efecto Doppler para medir las velocidades.

%No es lo mismo el movimiento en el centro galáctico que en el disco, en el disco se asume rotación pura, mientras que en el centro hay varios fenómenos como expansion, colapso, movimiento energetico, explosiones, etc.

%Las velocidades iguales a 0 (en el diagrama longitud-velocidad) es cuando las estrellas están en el círculo solar, por lo que se puede confundir con algo muy lejos.

%Parece que las coordenadas están en:
%\begin{gather*}
%300 ^{\circ}<l<348^{\circ}\\
%-2^{\circ}<b<2^{\circ}
%\end{gather*}

%======================================================

%Para medir la velocidad de las estrellas en la galaxia se ocupa como trazador del hidrógeno molecular, H\textsubscript{2}, al monóxido de carbono, CO, ya que es la segunda molécula más abundante, tiene un fuerte momento dipolar y sus transiciones son \textit{ubicuas}.
\begin{figure}[H]
    \centering
    \includegraphics[width=3in]{/Users/javier/Downloads/IMG_1699.pdf}
    \caption{Cinemática de una partícula orbitando el centro galácitco, cuyo movimiento es observado desde la Tierra.}
    \label{fig:mecanica}
\end{figure}

Siguiendo la Figura \ref{fig:mecanica} se tiene que la velocidad de una estrella, en la línea de visión viene dada por:
\begin{equation*}
v_{\shortparallel}(R)=v(R)cos(\alpha)-v(R_{\odot}sin(l))
\end{equation*}
donde se tiene la distancia del Sol al centro de la galaxia $R_{\odot}=8.5$ kpc, además por teorema del seno se tiene,
\begin{equation*}
v_{\shortparallel}(R)=v(R)\frac{R_{\odot}}{R}sin(l)-v(R_{\odot}sin(l))\\
\end{equation*}
\[
\Leftrightarrow v_{\shortparallel}(R)=[\omega(R)-\omega(R_{\odot})]R_{\odot}sin(l) \tag{2.1} \label{eq:ec-maestra}
\]
esta última ecuación se le conoce como \textit{la ecuación maestra}.

Haciendo un análisis de la velocidad de una partícula de masa $m$ a una distancia $R$ del centro galáctico, se iguala la fuerza gravitacional con la fuerza centrifuga, con lo que se tiene:
\[
m\frac{v^2(R)}{R}=G\frac{M(R)m}{R^2}
\]
\[
 \Rightarrow v(R)=\sqrt{G\frac{M(R)}{R}} \tag{2.2} \label{eq:velocidad}
\]
\[
 \Rightarrow \omega(R)=\frac{v(R)}{R}=\sqrt{G\frac{M(R)}{R^3}}.  \tag{2.3} \label{eq:omega}
\]
donde $v(R)$ es la velocidad de la partícula, $\omega(R)$ la velocidad angular, $G$ la constante de gravitación de Newton y $M(R)$ la masa de la galaxia encerrada dentro del radio $R$.

%\begin{gather*}
%m\frac{v^2(R)}{R}=G\frac{M(R)m}{R^2}\\
 %\Rightarrow v(R)=\sqrt{G\frac{M(R)}{R}}\\ \tag{2.2} \label{eq:velocidad}
  %\Rightarrow \omega(R)=\frac{v(R)}{R}=\sqrt{G\frac{M(R)}{R^3}}.  \tag{2.3} \label{eq:omega}
%\end{gather*}
Así que, si se tiene que la galaxia tiene una forma esférica $M(R) \propto R^3\Rightarrow\omega(R)$ es constante, pero si tiene forma de disco $M(R) \propto R^2\Rightarrow\omega(R)\propto \sqrt{1/R}$, por lo que $\omega$ es constante o disminuye con la distancia al centro. Con esto se puede calcular la velocidad máxima de $v_{\shortparallel}$. El máximo valor de  $\omega(R)$, que maximiza $v_{\shortparallel}$, se da para la menor distancia al centro galáctico que corresponde a $R_{min}=R_{\odot}sin(l)$, por lo que la velocidad máxima corresponde a la velocidad tangencial.


Para realizar esta tarea se ocupó un cubo de datos que contiene: Longitud, latitud y velocidad, donde para cada punto en el cielo $(l,b)$ se midió la velocidad máxima, o sea, la velocidad tangencial del cuerpo observado. Esta velocidad se escoge como la primera velocidad que tenga una temperatura de antena $T_A>5\sigma$ (la velocidad más grande en magnitud que corresponde a la señal y no ruido), midiendo desde las velocidades en magnitud más grandes a las más pequeñas, con $\sigma =0.1$ \si{\kelvin}. El procedimiento consistió en fijar las longitudes, $l$, y medir la velocidad que cumpla con la condición para la temperatura, que se guarda al igual que la latitud, $b$, correspondiente a la velocidad, además, se calcula el radio tangencial denotado por $R_T$ por geometría igual a $R_{\odot}sin(l)$. Con estos valores se calcula la velocidad angular $\omega(R)$ despejando de la ecuación maestra \ref{eq:ec-maestra} evaluada en $R_T$,
\begin{equation*}
\omega(R_{\odot}sin(l))=\frac{v_T(l)}{R_{\odot}sin(l)}+\omega_{\odot},
\end{equation*}
con $\omega_{\odot}$ la velocidad angular del Sol igual a 25.882 rads\textsuperscript{-1}. Esta velocidad angular se calcula para cada velocidad tangencial de cada longitud, con lo que se consigue el gráfico de $\omega(R)$ vs $R$, junto con el gráfico de la velocidad tangencial versus el radio, $v_T(R)$ vs $R$, ambos presentes en la Figura \ref{fig:curva-rotacion}.

Después \href{https://www.youtube.com/watch?v=8ISCwe7OM-8}{el profe }habla sobre la \textit{distancia cinemática}.

\section{Corrugación del plano}
El disco galáctico no es completamente plano, sino que tiene una pequeña corrugación que es mucho menor que el diámetro de la galaxia, aproximadamente la corrugación equivale a  $\sim 1\%$ del diámetro.

Esto se debe a que la galaxia tiene una estructura ondulatoria que depende del ángulo $\theta$ medido en el plano galáctico, y la distancia $R$ a su centro. Funciona como una onda espiral de densidad, y en $R$ hay modos normales de oscilación. %Se puede asimilar al parche de un tambor ya que se puede expresar analiticamente como una composicion de funciones de Bessel (nosotros lo hacemos más simple utilizando los datos) (agregar más de esto).

Para medir esta corrugación se utilizaron los mismo datos usados para medir la curva de rotación. Se toma como posición de la densidad máxima para cada longitud, como la posición en la latitud para la cual se consigue la velocidad máxima en magnitud asociada, $v_{T}$, que se llamará $b_{min}$ (se define como $min$ debido a que como se midió en el cuarto cuadrante, la velocidad máxima en magnitud equivale a la velocidad más negativa). En ese punto $(l,b_{min})$ se tiene:
\begin{gather*}
R=R_{\odot}sin(l) \ \land \ d=R_{\odot}cos(l),
\end{gather*}
por lo que, siguiendo la representación de la Figura (...), se consigue que la altura de dicho punto viene dada por:
\begin{equation*}
z=d\cdot tan(b_{min})=R_{\odot}cos(l)\cdot tan(b_{min}),
\end{equation*}
con lo que se puede calcular la altura de las posiciones para cada longitud y graficar la relación con la distancia $R=R_{\odot}sin(l)$, presente en la Figura \ref{fig:corrugacion}.

\section{Ajuste de modelo de masa}
Para encontrar una expresión analítica para la curva de rotación se debe ocupar un modelo físico definido por la distribución de la masa de la galaxia. Para esto se ocupa la ecuación (\ref{eq:velocidad}),
\begin{gather*}
v(R)=\sqrt{G\frac{M(R)}{R}}.
\end{gather*}

Para este informe se estudiaron 5 modelos de la distribución de masa $M(R)$:
\begin{itemize}
\item Modelos de un único parámetro libre
\begin{itemize}
\item Masa puntual en el centro galáctico: $M(R)=M_0$.
\item Disco uniforme: $M(R)=\pi R^2S$, con $S$ la densidad superficial homogénea del disco.
\item Esfera uniforme: $M(R)=\frac{4}{3}\pi R^3 \rho$, con $\rho$ la densidad volumétrica homogénea de la esfera.
\end{itemize}
\item Modelos de dos parámetros libres
\begin{itemize}
\item Disco uniforme más una masa puntual central: $M(R)=\pi R^2S+M_0$.
\item Esfera uniforme más una masa puntual central: $M(R)=\frac{4}{3}\pi R^3 \rho+M_0$
\end{itemize}
\end{itemize}

La distribuciones de masa de los distintos modelos se reemplazaron en la ecuación \ref{eq:velocidad} y en el script de Python se definieron las funciones \texttt{distribucion\_masa} que se utilizaron para realizar un fiteo de la curva de rotación obtenida de forma empírica. Los fiteos de cada modelo se presentan en los plots de la Figura \ref{fig:fit-curva-rotacion}, mientras que los valores de los parámetros de cada modelo, utilizados para el fit se muestran en la siguiente tabla:

\begin{table}[H]
\centering
\begin{tabular}{cccc}
\hline
Modelo                & \begin{tabular}[c]{@{}c@{}}M\textsubscript{0}\\ unidad\end{tabular} & \begin{tabular}[c]{@{}c@{}}S\\ Unidad\end{tabular} & \begin{tabular}[c]{@{}c@{}}$\rho$\\ unidad\end{tabular} \\ \hline
Masa puntual          & \num{3.228e10}                                      &                                                    &                                                      \\
Disco uniforme        &                                                     & \num{5.888e8}                                       &                                                      \\
Esfera uniforme       &                                                     &                                                    & \num{7.964e7}                                         \\
Disco + masa puntual  & \num{-3.403e8}                                       & \num{5.941e8}                                       &                                                      \\
Esfera + masa puntual & \num{7.981e9}                                        &                                                    & \num{6.372e7}                                         \\ \hline
\end{tabular}
\caption{Parámetros de cada modelo utilizados para el fiteo de la curva de rotación.}
\label{tab:parametros}
\end{table}


\section{Análisis y Conclusiones}

\section{Anexos}
El script utilizado para esta tarea se encuentra en \href{https://github.com/Wenupi/Astro_exp}{este repositorio de GitHub}.



\begin{figure}[H]
    \centering
    \includegraphics{/Users/javier/Documents/University/5th_semester/Experimental_astronomy/Astro_exp/Tarea2/codigo/img/curva_rotacion.pdf}
    \caption{Curvas de rotación para la velocidad tangencial en el gráfico de más arriba y para la velocidad angular en el gráfico de abajo, en función de la distancia al centro galáctico}
    \label{fig:curva-rotacion}
\end{figure}

\begin{figure}[H]
    \centering
    \includegraphics{/Users/javier/Documents/University/5th_semester/Experimental_astronomy/Astro_exp/Tarea2/codigo/img/corrugacion.pdf}
    \caption{Corrugación de la vía láctea para el cuarto cuadrante, representado como la altura con respecto al ecuador galáctico versus la distancia al centro galáctico.}
    \label{fig:corrugacion}
\end{figure}

\begin{figure}[H]
    \centering
    \includegraphics{/Users/javier/Documents/University/5th_semester/Experimental_astronomy/Astro_exp/Tarea2/codigo/img/fiteo_rotacion.pdf}
    \caption{Fiteos de la curva de rotación de la velocidad tangencial para el cuarto cuadrante de la vía láctea, utilizando los 5 modelos descritos en la sección 4.}
    \label{fig:fit-curva-rotacion}
\end{figure}



% FIN DEL DOCUMENTO
\end{document}
